\documentclass[tikz, border=0.5]{standalone}
%\documentclass{article}

\usepackage{tikz}
\usetikzlibrary{calc, patterns, intersections}

%%%%%%%%%%%%%%%%%%%%%%%%%%%%%%%%%%%%%%%
% note, 
%	0 - is an empty space
%	1 - is an other agent
%	2 - is an edge
%%%%%%%%%%%%%%%%%%%%%%%%%%%%%%%%%%%%%%%

% model parameters
\pgfmathtruncatemacro{\memory}{1}
\pgfmathtruncatemacro{\visibility}{3}
\pgfmathtruncatemacro{\obscure}{0}
\pgfmathsetmacro{\maxtablelength}{16}

\begin{document}
	
	\makeatletter
	\newcommand*\bigcdot{\mathpalette\bigcdot@{.5}}
	\newcommand*\bigcdot@[2]{\mathbin{\vcenter{\hbox{\scalebox{#2}{$\m@th#1\bullet$}}}}}
	\makeatother
	
	\def\noAgent{0}%
	\def\agent{1}%
	\def\leftAppend{0}%
	\def\rightAppend{1}%
	\def\edge{2}%
	\protected\def\Pzero{\phantom{\noAgent}}%

	\newcommand{\AppendToStrBefore}[2]
	{
		\ifnum#2=\leftAppend%
			\xdef\str{\str#1}%
		\else%
			\xdef\str{#1\str}%
		\fi%
	}
	
	\newcommand{\AppendToStrAfter}[2]
	{
		\ifnum#2=\leftAppend%
			\xdef\str{#1\str}%
		\else%
			\xdef\str{\str#1}%
		\fi%
	}
	
	\newcommand{\CreateVStr}[2]
	{%
		\xdef\str{}%
		\ifnum#1=1%
			\xdef\str{\agent}%
		\else%
			\ifnum#1<\onesidestates%
				\pgfmathtruncatemacro{\lastseen}{mod(#1, 2) + \agent}%
				\xdef\str{\lastseen}%
			\fi%
			\pgfmathtruncatemacro{\zeros}{#1 / 2}%
			\foreach \zero in {1, ..., \zeros}%
			{%
				\AppendToStrBefore{\noAgent}{#2}
			}%
		\fi%
		\pgfmathtruncatemacro{\emptyspaces}{\visibility - int(#1 / 2) - 1}%
		\ifnum\emptyspaces>0%
			\foreach \zero in {1, ..., \emptyspaces}%
			{%
				\AppendToStrAfter{\Pzero}{#2}
			}%
		\fi%
	}%

	\newcommand{\CreateLeftStr}[1]
	{%
		\CreateVStr{#1}{\leftAppend}%
		\xdef\leftstr{\str}%
	}%
	
	\newcommand{\CreateRightStr}[1]
	{%
		\CreateVStr{#1}{\rightAppend}%
		\xdef\rightstr{\str}%
	}%

	\newcommand{\CreateNStr}[3]
	{
		\xdef\str{}%
		\foreach \i[evaluate=\i as \modx using 2^(\i+1)] in {0, ..., #1} % while loop
		{
			\ifnum#1>\i%
				\pgfmathtruncatemacro{\modremainder}{mod(#2, \modx)}%
				\pgfmathtruncatemacro{\modx}{\modx / 2}
				\ifnum\modx>0%
					\pgfmathtruncatemacro{\digit}{\modremainder / \modx}%
				\else%
					\pgfmathtruncatemacro{\digit}{\modremainder}%
				\fi%
				\AppendToStrBefore{\digit}{#3}%
			\else%
				\breakforeach%
			\fi%
		}
		\ifnum#1<\visibility%
			\AppendToStrAfter{\edge}{#3}%
			\pgfmathtruncatemacro{\invisibleLength}{\visibility - #1 - 1}
			\ifnum\invisibleLength>0%
				\foreach \invisiblespace in {1,...,\invisibleLength}
				{
					\AppendToStrAfter{\Pzero}{#3}%
				}
			\fi%
		\fi%
	}

	\newcommand{\CreateNLeftStr}[2]
	{
		\CreateNStr{#1}{#2}{\leftAppend}%
		\xdef\leftStr{\str}%
	}
	\newcommand{\CreateNRightStr}[2]
	{
		\CreateNStr{#1}{#2}{\rightAppend}%
		\xdef\rightStr{\str}%
	}
	
	% drawing parameters
	\pgfmathsetmacro{\lengthlabelline}{0.8}
	\pgfmathsetmacro{\labelxoffset}{0.5}
	\pgfmathsetmacro{\labelyoffset}{0.1}
	\pgfmathsetmacro{\tiklabelxoffset}{0.5}
	\pgfmathsetmacro{\tiklabelyoffset}{0.3}
	\pgfmathsetmacro{\cellsize}{1}
	\pgfmathsetmacro{\above}{0.5+\visibility/20}
	\pgfmathsetmacro{\scalefactor}{0.8}
	
	\begin{tikzpicture}
		% common code
		\pgfmathtruncatemacro{\tableheight}{2^(\memory)}
		\pgfmathtruncatemacro{\onesidestates}{ifthenelse(\obscure == 1, 2 * \visibility, 2 ^ (\visibility + 1) - 1)}
		\pgfmathtruncatemacro{\numberofstates}{\onesidestates ^ 2}
		\pgfmathtruncatemacro{\numberoftables}{ceil(\numberofstates / \maxtablelength)}
		
		\begin{scope}[scale=\cellsize]
			% draw all the tables
			\foreach \tablenum in {1, ..., \numberoftables}
			{
				\pgfmathtruncatemacro{\tablelength}{ifthenelse(\tablenum < \numberoftables, \maxtablelength, \numberofstates - (\numberoftables - 1) * \maxtablelength)}
				\pgfmathsetmacro{\yoffset}{-(\tablenum - 1) * (\tableheight + 1.5)}
				
				\begin{scope}[yshift=\yoffset cm]
					\coordinate (left_upper_corner_\tablenum) at (0, \tableheight);
					\draw (0, 0) grid (\tablelength, \tableheight);
					\begin{scope}[shift={(left_upper_corner_\tablenum)}]
						\draw (0, 0) -- ++(135:\lengthlabelline);
						\node[rotate=-45,scale=0.5] at ($(135:\labelxoffset)+(45:\labelyoffset)$) {input};
						\node[rotate=-45,scale=0.5] at ($(135:\labelxoffset)+(180+45:\labelyoffset)$) {memory};
					\end{scope}
					\foreach \state in { 1, ..., \tableheight}
					{
						\pgfmathtruncatemacro{\memorystate}{\state-1}
						\node (memory_\tablenum_\state) at (0 - \tiklabelyoffset, \tableheight - \state + \tiklabelxoffset) {$\memorystate$};
					}					
				\end{scope}
			}
			
			\ifdim\obscure pt=1 pt%
				% prepare obscured visible state in one dimension :left
				\foreach \leftstate in {1, ..., \onesidestates}
				{
					\CreateLeftStr{\leftstate}
					
					% right part of visible state
					\foreach \rightstate in {1, ..., \onesidestates}
					{
						\CreateRightStr{\rightstate}

						\pgfmathtruncatemacro{\currentstate}{(\leftstate - 1) * \onesidestates + (\rightstate - 1)}
						\pgfmathtruncatemacro{\tablenum}{ceil((\currentstate + 1) / \maxtablelength)}
						\pgfmathtruncatemacro{\columnnum}{mod(\currentstate, \maxtablelength)}
						\begin{scope}[shift={(left_upper_corner_\tablenum)}]
							\node[above=\above cm, rotate=90, scale=\scalefactor] at (\columnnum + \scalefactor - 0.1, 0) {$\leftstr\bigcdot\rightstr$};
						\end{scope}
					}
				}
			\else%		
				% prepare non-obscured visible state : left
				\foreach \kleft in {0,...,\visibility}
				{
					\pgfmathtruncatemacro{\leftkstates}{2^\kleft - 1}
					\foreach \jleft in {0,...,\leftkstates}
					{
						\CreateNLeftStr{\kleft}{\jleft}
						
						\foreach \kright in {0,...,\visibility}
						{
							\pgfmathtruncatemacro{\rightkstates}{2^\kright - 1}
							\foreach \jright in {0,...,\rightkstates}
							{
								\CreateNRightStr{\kright}{\jright}
								
								\pgfmathtruncatemacro{\currentstate}{(2^\kleft + \jleft - 1) * \onesidestates + (2^\kright + \jright - 1)}
								\pgfmathtruncatemacro{\tablenum}{ceil((\currentstate + 1) / \maxtablelength)}
								\pgfmathtruncatemacro{\columnnum}{mod(\currentstate, \maxtablelength)}
								
								\begin{scope}[shift={(left_upper_corner_\tablenum)}]
									\node[above=\above cm, rotate=90, scale=\scalefactor] at (\columnnum + \scalefactor - 0.1, 0) {$\leftStr\bigcdot\rightStr$};
								\end{scope}
							}
						}						
					}
				}
			\fi%

		\end{scope}
	\end{tikzpicture}
\end{document}
