\documentclass[tikz, border=0.5]{standalone}

\usepackage{tikz}
\usepackage{fmtcount}% http://ctan.org/pkg/fmtcount
\usetikzlibrary{calc, patterns, intersections}

%%%%%%%%%%%%%%%%%%%%%%%%%%%%%%%%%%%%%%%
% note, 
%	0 - is an empty space; left uncolored
%	1 - is an other agent; filled with black
%	2 - is an edge; cross-hatched
%	3 - is an unspecified (obscured location); filled with gray
%%%%%%%%%%%%%%%%%%%%%%%%%%%%%%%%%%%%%%%

% model parameters
\pgfmathtruncatemacro{\memory}{2}
\pgfmathtruncatemacro{\visibility}{2}
\pgfmathsetmacro{\maxtablelength}{8}

\begin{document}
	
	\newcommand*{\ExtractCoordinate}[1]{\path (#1); \pgfgetlastxy{\XCoord}{\YCoord}}%
	
	% calculate the sizes of the neighborhood
	\newcommand{\GetNeighborhoodSizes}[1]{%
		\xdef\toleft{0}
		\xdef\toright{0}
		\xdef\toup{0}
		\xdef\todown{0}
		
		\pgfmathtruncatemacro{\tableSize}{2 * \visibility + 1}
		\foreach \columnoffset/\rowoffset[count=\i] in #1 {%
			\ifnum\columnoffset<\toleft%
				\xdef\toleft{\columnoffset}%
			\fi%
			\ifnum\columnoffset>\toright%
				\xdef\toright{\columnoffset}%
			\fi%
			\ifnum\rowoffset<\todown%
				\xdef\todown{\rowoffset}%
			\fi%
			\ifnum\rowoffset>\toup%
				\xdef\toup{\rowoffset}%
			\fi%
			
			\coordinate (anchor_\i) at (\columnoffset, \rowoffset);
		}
		
		\xdef\xbase{\toleft}
		\xdef\ybase{\todown}
		\pgfmathtruncatemacro{\pxsize}{\toright - \toleft + 1}
		\xdef\xsize{\pxsize}
		\pgfmathtruncatemacro{\pysize}{\toup - \todown + 1}
		\xdef\ysize{\pysize}
		
	}	
	
	% draw this neighborhood
	\newcommand{\DrawNeighborhood}[1]{%
		\begin{scope}[shift={(-\xbase, -\ybase)}]		
			\draw[fill=black] (0.5, + 0.5) circle (3pt);
			\draw (0, 0) rectangle +(1,1);
			
			
			\foreach \value[count=\i] in #1 {%
				\ExtractCoordinate{anchor_\i}
				\begin{scope}[shift={(\XCoord, \YCoord)}]
					\ifnum\value=0%
						\draw (0, 0) rectangle +(1, 1);
					\fi%
					\ifnum\value=1%
						\draw[fill=black] (0, 0) rectangle +(1, 1);
					\fi%
					\ifnum\value=2%
						\draw[pattern=north west lines] (0, 0) rectangle +(1, 1);
					\fi%
					\ifnum\value=3%
						\draw[fill=black!10] (0, 0) rectangle +(1, 1);
						\node at (0.5, 0.5) {\bf ?};
					\fi%
				\end{scope}
			}
		\end{scope}
	}

	% set column and table num
	\newcommand{\DrawAtIndex}[1]{%
		\pgfmathtruncatemacro{\currentstate}{#1}
		\pgfmathtruncatemacro{\tablenum}{ceil((\currentstate + 1) / \maxtablelength)}
		\pgfmathtruncatemacro{\columnnum}{mod(\currentstate, \maxtablelength)}
		
		\pgfmathsetmacro{\startoffset}{(1 - \xsize * \scalefactor) / 2}
		\begin{scope}[shift={($(left_upper_corner_\tablenum) + (\columnnum + \startoffset, 0)$)}, scale=\scalefactor]
			\DrawNeighborhood{\neighbors}
		\end{scope}
	}
	
	% drawing parameters
	\pgfmathsetmacro{\lengthlabelline}{0.8}
	\pgfmathsetmacro{\labelxoffset}{0.5}
	\pgfmathsetmacro{\labelyoffset}{0.1}
	\pgfmathsetmacro{\tiklabelxoffset}{0.5}
	\pgfmathsetmacro{\tiklabelyoffset}{0.3}
	\pgfmathsetmacro{\mintablespace}{0.7}
	\pgfmathsetmacro{\cellsize}{1}
	\pgfmathsetmacro{\above}{0.5+\visibility/20}
	\pgfmathsetmacro{\scalefactor}{0.4}
	
	\begin{tikzpicture}
		% common code
		\pgfmathtruncatemacro{\tableheight}{2^(\memory)}
		\pgfmathtruncatemacro{\onesidestates}{2 * \visibility}
		\pgfmathtruncatemacro{\numberofstates}{\onesidestates ^ 2}
		\pgfmathtruncatemacro{\numberoftables}{ceil(\numberofstates / \maxtablelength)}
		
		\newcommand{\neighborsCoordinates}{-1/1, 0/2, -1/0, 0/1}
		\newcommand{\neighbors}{}
		\GetNeighborhoodSizes{\neighborsCoordinates}
		
		\begin{scope}[scale=\cellsize]
			% draw all the tables
			\foreach \tablenum in {1, ..., \numberoftables}
			{
				\pgfmathtruncatemacro{\tablelength}{ifthenelse(\tablenum < \numberoftables, \maxtablelength, \numberofstates - (\numberoftables - 1) * \maxtablelength)}
				\pgfmathsetmacro{\yoffset}{-(\tablenum - 1) * (\tableheight + \mintablespace + \scalefactor * \ysize)}
				
				\begin{scope}[yshift=\yoffset cm]
					\coordinate (left_upper_corner_\tablenum) at (0, \tableheight);
					\draw (0, 0) grid (\tablelength, \tableheight);
					\begin{scope}[shift={(left_upper_corner_\tablenum)}]
						\draw (0, 0) -- ++(135:\lengthlabelline);
						\node[rotate=-45,scale=0.5] at ($(135:\labelxoffset)+(45:\labelyoffset)$) {input};
						\node[rotate=-45,scale=0.5] at ($(135:\labelxoffset)+(180+45:\labelyoffset)$) {memory};
					\end{scope}
					\foreach \state in { 1, ..., \tableheight}
					{
						\pgfmathtruncatemacro{\memorystate}{\state-1}
						\node (memory_\tablenum_\state) at (0 - \tiklabelyoffset, \tableheight - \state + \tiklabelxoffset) {$\padzeroes[\memory]{\binarynum{\memorystate}}$};
					}					
				\end{scope}
			}
			
			\renewcommand{\neighbors}{2, 1, 0, 3}			
			\DrawAtIndex{0}
			\renewcommand{\neighbors}{1, 1, 0, 0}			
			\DrawAtIndex{1}

		\end{scope}
	\end{tikzpicture}
\end{document}
