\documentclass[tikz]{standalone}

\usepackage{amsmath, amssymb}
\usepackage{fmtcount}% http://ctan.org/pkg/fmtcount
\usetikzlibrary{calc, patterns}

%%%%%%%%%%%%%%%%%%%%%%%%%%%%%%%%%%%%%%%%
%
% note, 
%	0 - is an empty space; left uncolored
%	1 - is an other agent; filled with black
%	2 - is an edge; cross-hatched
%	3 - is a do not care location
%
%%%%%%%%%%%%%%%%%%%%%%%%%%%%%%%%%%%%%%%%

\begin{document}

	% model parameters
	\pgfmathtruncatemacro{\memory}{2}%
	\pgfmathtruncatemacro{\visibility}{2}%
	\pgfmathtruncatemacro{\maxtablelength}{8}%

	% calculate the sizes of the neighborhood
	\newcommand{\GetNeighborhoodSizes}[1]{%
		\xdef\toleft{0}
		\xdef\toright{0}
		\xdef\toup{0}
		\xdef\todown{0}

		\pgfmathtruncatemacro{\tableSize}{2 * \visibility + 1}
		\foreach \columnoffset/\rowoffset[count=\i] in #1 {%
			\ifnum\columnoffset<\toleft%
				\xdef\toleft{\columnoffset}%
			\fi%
			\ifnum\columnoffset>\toright%
				\xdef\toright{\columnoffset}%
			\fi%
			\ifnum\rowoffset<\todown%
				\xdef\todown{\rowoffset}%
			\fi%
			\ifnum\rowoffset>\toup%
				\xdef\toup{\rowoffset}%
			\fi%
			\coordinate (anchor_\i) at (\columnoffset, \rowoffset);
		}%

		\xdef\xbase{\toleft}
		\xdef\ybase{\todown}
		\pgfmathtruncatemacro{\pxsize}{\toright - \toleft + 1}
		\xdef\xsize{\pxsize}
		\pgfmathtruncatemacro{\pysize}{\toup - \todown + 1}
		\xdef\ysize{\pysize}

	}%

	\newcommand*{\ExtractCoordinate}[1]{\path[overlay] (#1); \pgfgetlastxy{\XCoord}{\YCoord}}%

	% draw this neighborhood
	\newcommand{\DrawNeighborhood}[1]{%
		\begin{scope}[scale=\scalefactor]
			\begin{scope}[shift={(-\xbase, -\ybase)}]
				\draw[fill=black] (0.5, 0.5) circle (3pt);
				\draw (0, 0) rectangle (1,1);

				\foreach \value[count=\i] in #1 {%
					\ExtractCoordinate{anchor_\i}
					\begin{scope}[shift={(\XCoord, \YCoord)}]
						\ifnum\value=0%
							\draw (0, 0) rectangle (1, 1);
						\fi%
						\ifnum\value=1%
							\draw[fill=black!50] (0, 0) rectangle (1, 1);
						\fi%
						\ifnum\value=2%
							\begin{scope}[opacity=\opacityfactor, blend group=normal]
								\draw[pattern=north west lines, draw = none] (0, 0) rectangle (1, 1);
							\end{scope}
							\draw (0, 0) rectangle (1, 1);
						\fi%
						\ifnum\value=3%
							\begin{scope}[opacity=\opacityfactor]
								\clip (0, 0) rectangle (1, 1);
								\node[transform shape] at (0.5, 0.5) {\Large $\mathbf{X}$};
							\end{scope}
							\draw (0, 0) rectangle (1, 1);
						\fi%
					\end{scope}
				}%
			\end{scope}
		\end{scope}
	}%

	%% set column and table num
	\newcommand{\DrawAtIndex}[1]{%
		% local parameters
		\pgfmathtruncatemacro{\currentstate}{#1}%
		\pgfmathtruncatemacro{\tablenum}{ceil((\currentstate + 1) / \maxtablelength)}%
		\pgfmathtruncatemacro{\columnnum}{mod(\currentstate, \maxtablelength)}%

		\pgfmathsetmacro{\startoffset}{(1 - \xsize * \scalefactor) / 2}
		\begin{scope}[shift={($(left_upper_corner_\tablenum) + (\columnnum + \startoffset, 0)$)}]
			\DrawNeighborhood{\neighbors}
		\end{scope}
	}%

	% put values into the table
	\newcommand{\SetTableValues}[4]{%
		% local parameters
		\pgfmathtruncatemacro{\currentstate}{#1}%
		\pgfmathtruncatemacro{\tablenum}{ceil((\currentstate + 1) / \maxtablelength)}%
		\pgfmathtruncatemacro{\columnnum}{mod(\currentstate, \maxtablelength)}%

		\xdef\pickedDirection{\text{,}}
		\ifnum#4=0%
			\xdef\pickedDirection{\pickedDirection\text{N}}%
		\fi%
		\ifnum#4=1%
			\xdef\pickedDirection{\pickedDirection\text{E}}%
		\fi%
		\ifnum#4=2%
			\xdef\pickedDirection{\pickedDirection\text{S}}%
		\fi%
		\ifnum#4=3%
			\xdef\pickedDirection{\pickedDirection\text{W}}%
		\fi%
		\ifnum#4=4%
			\xdef\pickedDirection{\pickedDirection\varnothing}%
		\fi%
		\begin{scope}[shift={($(left_upper_corner_\tablenum) + (\columnnum, -#2)$)}]
			\node at (\tiklabelxoffset, \tiklabelyoffset) {\small $\padzeroes[\memory]{\binarynum{#3}}\pickedDirection$};
		\end{scope}
	}%

	% drawing parameters
	\pgfmathsetmacro{\scalefactor}{0.3}%
	\pgfmathsetmacro{\opacityfactor}{0.7}%
	\pgfmathsetmacro{\tiklabelxoffset}{0.5}%
	\pgfmathsetmacro{\tiklabelyoffset}{0.5}%
	\pgfmathsetmacro{\lengthlabelline}{0.8}%
	\pgfmathsetmacro{\labelxoffset}{0.5}%
	\pgfmathsetmacro{\labelyoffset}{0.1}%
	\pgfmathsetmacro{\mintablespace}{0.7}%
	\pgfmathsetmacro{\cellsize}{1}%

	\begin{tikzpicture}
		% common code
		\pgfmathtruncatemacro{\numberofstates}{16}%
		\pgfmathtruncatemacro{\tableheight}{2^(\memory)}%
		\pgfmathtruncatemacro{\numberoftables}{ceil(\numberofstates / \maxtablelength)}%

		\newcommand{\neighborsCoordinates}{0/2, 0/1, 0/-1, 0/-2, 1/0, -1/0, 1/1, 2/0, 1/-1, -1/-1, -2/0, -1/1}
		\newcommand{\neighbors}{}
		\GetNeighborhoodSizes{\neighborsCoordinates}

		\begin{scope}[scale=\cellsize]
			% draw all the tables
			\foreach \tablenum in {1, ..., \numberoftables}{%
				\pgfmathtruncatemacro{\tablelength}{ifthenelse(\tablenum < \numberoftables, \maxtablelength, \numberofstates - (\numberoftables - 1) * \maxtablelength)}%
				\pgfmathsetmacro{\yoffset}{-(\tablenum - 1) * (\tableheight + \mintablespace + \scalefactor * \ysize)}%

				\begin{scope}[yshift=\yoffset cm]
					\coordinate (left_upper_corner_\tablenum) at (0, \tableheight);
					\draw (0, 0) grid (\tablelength, \tableheight);
					\begin{scope}[shift={(left_upper_corner_\tablenum)}]
						\draw (0, 0) -- ++(135:\lengthlabelline);
						\node[rotate=-45,scale=0.5] at ($(135:\labelxoffset)+(45:\labelyoffset)$) {input};
						\node[rotate=-45,scale=0.5] at ($(135:\labelxoffset)+(180+45:\labelyoffset)$) {memory};
					\end{scope}
					\foreach \state in { 1, ..., \tableheight}{%
						\pgfmathtruncatemacro{\memorystate}{\state-1}
						\node (memory_\tablenum_\state) at (-1 + \tiklabelxoffset, \tableheight - \state + \tiklabelyoffset) {$\padzeroes[\memory]{\binarynum{\memorystate}}$};
					}%
				\end{scope}
			}%

			\renewcommand{\neighbors}{3, 1, 1, 3}
			\DrawAtIndex{0}
			\renewcommand{\neighbors}{1, 0, 1, 3}
			\DrawAtIndex{1}
			\renewcommand{\neighbors}{2, 0, 1, 3}
			\DrawAtIndex{2}
			\renewcommand{\neighbors}{0, 0, 1, 3}
			\DrawAtIndex{3}
			\renewcommand{\neighbors}{3, 1, 0, 1}
			\DrawAtIndex{4}
			\renewcommand{\neighbors}{1, 0, 0, 1}
			\DrawAtIndex{5}
			\renewcommand{\neighbors}{2, 0, 0, 1}
			\DrawAtIndex{6}
			\renewcommand{\neighbors}{0, 0, 0, 1}
			\DrawAtIndex{7}
			\renewcommand{\neighbors}{3, 1, 0, 2}
			\DrawAtIndex{8}
			\renewcommand{\neighbors}{1, 0, 0, 2}
			\DrawAtIndex{9}
			\renewcommand{\neighbors}{2, 0, 0, 2}
			\DrawAtIndex{10}
			\renewcommand{\neighbors}{0, 0, 0, 2}
			\DrawAtIndex{11}
			\renewcommand{\neighbors}{3, 1, 0, 0}
			\DrawAtIndex{12}
			\renewcommand{\neighbors}{1, 0, 0, 0}
			\DrawAtIndex{13}
			\renewcommand{\neighbors}{2, 0, 0, 0}
			\DrawAtIndex{14}
			\renewcommand{\neighbors}{0, 0, 0, 0}
			\DrawAtIndex{15}

			\SetTableValues{0}{1}{1}{4}
			\SetTableValues{1}{1}{3}{4}
			\SetTableValues{2}{1}{1}{4}
			\SetTableValues{3}{1}{3}{4}
			\SetTableValues{4}{1}{1}{2}
			\SetTableValues{5}{1}{1}{2}
			\SetTableValues{6}{1}{1}{2}
			\SetTableValues{7}{1}{1}{2}
			\SetTableValues{8}{1}{1}{2}
			\SetTableValues{9}{1}{1}{2}
			\SetTableValues{10}{1}{1}{2}
			\SetTableValues{11}{1}{1}{2}
			\SetTableValues{12}{1}{1}{2}
			\SetTableValues{13}{1}{1}{2}
			\SetTableValues{14}{1}{1}{2}
			\SetTableValues{15}{1}{1}{2}
			\SetTableValues{0}{2}{0}{4}
			\SetTableValues{1}{2}{0}{4}
			\SetTableValues{2}{2}{0}{4}
			\SetTableValues{3}{2}{0}{4}
			\SetTableValues{4}{2}{0}{4}
			\SetTableValues{5}{2}{0}{4}
			\SetTableValues{6}{2}{0}{4}
			\SetTableValues{7}{2}{0}{4}
			\SetTableValues{8}{2}{0}{4}
			\SetTableValues{9}{2}{0}{4}
			\SetTableValues{10}{2}{0}{4}
			\SetTableValues{11}{2}{0}{4}
			\SetTableValues{12}{2}{0}{4}
			\SetTableValues{13}{2}{0}{4}
			\SetTableValues{14}{2}{0}{2}
			\SetTableValues{15}{2}{0}{2}
			\SetTableValues{0}{3}{3}{4}
			\SetTableValues{1}{3}{3}{4}
			\SetTableValues{2}{3}{1}{4}
			\SetTableValues{3}{3}{3}{0}
			\SetTableValues{4}{3}{1}{4}
			\SetTableValues{5}{3}{3}{4}
			\SetTableValues{6}{3}{1}{4}
			\SetTableValues{7}{3}{3}{0}
			\SetTableValues{8}{3}{1}{4}
			\SetTableValues{9}{3}{3}{4}
			\SetTableValues{10}{3}{1}{4}
			\SetTableValues{11}{3}{3}{0}
			\SetTableValues{12}{3}{1}{4}
			\SetTableValues{13}{3}{3}{4}
			\SetTableValues{14}{3}{1}{4}
			\SetTableValues{15}{3}{3}{0}
			\SetTableValues{0}{4}{2}{4}
			\SetTableValues{1}{4}{2}{0}
			\SetTableValues{2}{4}{0}{4}
			\SetTableValues{3}{4}{2}{0}
			\SetTableValues{4}{4}{0}{4}
			\SetTableValues{5}{4}{2}{0}
			\SetTableValues{6}{4}{0}{4}
			\SetTableValues{7}{4}{2}{0}
			\SetTableValues{8}{4}{0}{4}
			\SetTableValues{9}{4}{2}{0}
			\SetTableValues{10}{4}{0}{4}
			\SetTableValues{11}{4}{2}{0}
			\SetTableValues{12}{4}{0}{4}
			\SetTableValues{13}{4}{2}{0}
			\SetTableValues{14}{4}{0}{4}
			\SetTableValues{15}{4}{2}{0}

		\end{scope}
	\end{tikzpicture}
\end{document}
